\documentclass[titlepage]{article}
\usepackage{array}
\usepackage{enumerate}
\usepackage{graphicx}
\usepackage{listings}

\begin{document}

\author{Stevan Stanisic and Santana Mach}
\title{COMP 8505 - Final Project \\ Rootkit \\ Design Documents}
\date{Nov 06, 2011}
\maketitle{}

\tableofcontents
\pagebreak

\section{Introduction}

Need an intro...

\section{Program Functionality}

The program will be made up of two executables, a Command and Control Client application and a Backdoor Server application.

The Client should have two modes, of which one will be active at a time:
\begin{itemize}
  \item Command Mode - In which the user can send commands to the Server and optionally receive responses.
  \item Exfiltration Mode - In which the user can update the watch list on the Server.
\end{itemize}

In both modes, the Client will also be listening for Exfiltration transmissions from the server and saving them to disk.

The Server will simultaneously be monitoring its network interfaces for incoming commands from the Client and its list of
watch directories for file changes.

\begin{lstlisting}
Pseudo-code for Client Program
{
  Main Function
  {
    Parse arguments;
    Read config file;

    Pass address into backdoor_client();
  }

  backdoor_client Function
  {
    Verify address and port;

    Setup listening_thread();

    While read stdin
    {
      if exfil switch is set
        set exfil path to send;

      Prepare raw socket;
    }
  }

  listen_thread Function
  {
    Prepare listening socket;

    While socket is open
    {
      Receive packet;
      Decrypt payload;
      Print decrypted payload;
    }
  }
}
\end{lstlisting}

\clearpage

\begin{lstlisting}
Pseudo-code for Server Program
{
  Main Function
  {
    Parse arguments;
    Read config file;

    Mask the process name;

    Pass filter into pcap_init();

    Execute srv_listen();
  }

  pcap_init Function
  {
    Open pcap packet capture;
    Build packet filter;
    Set packet filer;
  }

  srv_listen Function
  {
    Forever Loop
    {
      Process packet and pass into pkt_handler;
    }
  }

  pkt_handler Function
  {
    Locate payload portion of packet;

    Authenticate backdoor header key;

    Return if authentication fails;

    Decrypt payload with DES;

    Verify decrypted contents;

    Extract command from contents;

    Grab return address;

    Pass command and address into execute();
  }

  execute Function
  {
    Run command and grab standard out;

    Prepare raw socket for duplex;

    While content in stdout
    {
      Switch with covert mode;
      Encrypt content with DES;
      Send encrypted content to client;
    }   
  }

  covert_udplen Function
  {
    
  }

  covert_ntp Function
  {
    
  }

  covert_icmp Function
  {
    
  }

  exfil Function
  {

  }
}
\end{lstlisting}

\clearpage

\section{Communication Details}

\subsection{State Machine}

The communication between the client and server have to be reliable

\begin{figure}[htb]                                                                       
  \begin{center}
    \includegraphics[width=0.9\textwidth]{imgs/std.png}
  \end{center}
  \caption{Program State Transition Diagram}
  \label{fig:std}
\end{figure}

\begin{figure}[htb]                                                                       
  \begin{center}
    \includegraphics[width=0.5\textwidth]{imgs/comm.png}
  \end{center}
  \caption{Transmission State Transition Diagram}
  \label{fig:comm}
\end{figure}

\begin{lstlisting}
Pseudo-code for State Machine
{
  Send Machine
  {
    Prepare raw socket;
    
    Wait for command to trigger;
    
    While more to send
    {
      Send packet to receiving machine;
       
      Sleep to slow data stream;
    }
  }
  
  Receive Machine
  {
    Open pcap packet reading;
    
    Wait for trigger
    {
      Process packet;
      
      If last packet
      {
        If hash is incorrect
          Ask for retransmission;
        
        If server
          Send response or file;
      
        If client
          Display to command line;
      }
    }
  }
}
\end{lstlisting}

\clearpage

\subsection{Covert Channel}

\begin{figure}[htb]                                                                       
  \begin{center}
    \includegraphics[width=0.9\textwidth]{imgs/packet.png}
  \end{center}
  \caption{Packet Data Diagram}
  \label{fig:packet}
\end{figure}

\begin{figure}[htb]                                                                       
  \begin{center}
    \includegraphics[width=0.9\textwidth]{imgs/frame.png}
  \end{center}
  \caption{Control Frame Diagram}
  \label{fig:frame}
\end{figure}

\begin{figure}[htb]                                                                       
  \begin{center}
    \includegraphics[width=0.9\textwidth]{imgs/transmission.png}
  \end{center}
  \caption{Overall Transmission Diagram}
  \label{fig:transmission}
\end{figure}

\clearpage

\section{Task Breakdown}

\begin{itemize}
	\item Extract old code [Santana]
	\item Reliability
	\subitem Retransmissions [Steve]
	\subitem Control Packets [Steve]
	\subitem Timeouts [Steve]
	\item Covert Raw Socket Crafting
	\subitem UDP Length Mode [Santana]
	\subitem NTP Mode [Santana]
	\subitem ICMP Mode [Santana]
	\item Exfiltration
	\subitem Client [Steve]
	\subitem Server [Santana]
\end{itemize}

\clearpage

\section{Conclusion}

\end{document}
